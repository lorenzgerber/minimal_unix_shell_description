\documentclass[a4paper,11pt,twoside]{article}
%\documentclass[a4paper,11pt,twoside,se]{article}

\usepackage{UmUStudentReport}
\usepackage{verbatim}   % Multi-line comments using \begin{comment}
\usepackage{courier}    % Nicer fonts are used. (not necessary)
\usepackage{pslatex}    % Also nicer fonts. (not necessary)
\usepackage[pdftex]{graphicx}   % allows including pdf figures
\usepackage{listings}
\usepackage{pgf-umlcd}
%\usepackage{lmodern}   % Optional fonts. (not necessary)
%\usepackage{tabularx}
%\usepackage{microtype} % Provides some typographic improvements over default settings
%\usepackage{placeins}  % For aligning images with \FloatBarrier
%\usepackage{booktabs}  % For nice-looking tables
%\usepackage{titlesec}  % More granular control of sections.

% DOCUMENT INFO
% =============
\department{Department of Computing Science}
\coursename{System Level Programming 7.5 p}
\coursecode{5DV088}
\title{Minimal Shell - MISH}
\author{Lorenz Gerber ({\tt{dv15lgr@cs.umu.se}} {\tt{lozger03@student.umu.se}})}
\date{2016-10-07}
%\revisiondate{2016-01-18}
\instructor{Mikael Ränner / Filip Åberg / Jonathan Westin / Mattias Åsander}


% DOCUMENT SETTINGS
% =================
\bibliographystyle{plain}
%\bibliographystyle{ieee}
\pagestyle{fancy}
\raggedbottom
\setcounter{secnumdepth}{2}
\setcounter{tocdepth}{2}
%\graphicspath{{images/}}   %Path for images

\usepackage{float}
\floatstyle{ruled}
\newfloat{listing}{thp}{lop}
\floatname{listing}{Listing}



% DEFINES
% =======
%\newcommand{\mycommand}{<latex code>}

% DOCUMENT
% ========
\begin{document}
\lstset{language=C}
\maketitle
\thispagestyle{empty}
\newpage
\tableofcontents
\thispagestyle{empty}
\newpage

\clearpage
\pagenumbering{arabic}

\section{Problem Description} 
The aim of this laboration was to develop a minimal unix shell. The shell had to implement input and output stream redirection, stream piping between commands, and two internal commands, `cd' and `echo'. The main techniques to be used for implementing the shell were `pipes', `forking' and execution of programs by the `exec' function family. Further, the interrupt signal (SIGINT) had to be reassigned to a function that allows to stop all child processes and then returns control back to the shell prompt.

\section{Compilation and Invocation}
The present code uses safe signaling by `sigaction' which requires on linux to compile with std=gnu11 instead of ANSI C standard. On OSX, the code built also when using std=c11, hence the gcc included in linux seems to have a more stringent interpretation of `c11' standard (`sigaction' is POSIX, not ANSI C). A makefile is provided that takes care of building and linking the source by invoking `make all'. A `make clean' script is also included. 

\section{Detailed Usage Description}
`mish' does not take any command line argument. The shell can be quit by sending EOF to the prompt (Ctrl-D). SIGINT (Ctrl-C) is ignored in the prompt main loop and assigned to a custom function during exectution of external commands. The custom implementation stops all started child processes and returns back to the command prompt. 

\section{Algorithm Description}


\begin{verbatim}

loop over all but the last commands 
    create a pipe
    spawn a child process

    Child process code:
        if requested 
            setup redirect
        else
            close pipe read_end
        if `in' != 0
            dup2 `in' to STDIN
            close `in'
        dup2 pipe write_end to STDOUT
        close pipe write_end
        run command

    Parent process code:
        if `in' != 0, close 'in'
        set 'in' to pipe read_end 
        close pipe write_end

spawn a child process for last command

Child process code:
    if there is only one command and this command has 
        to redirect input from a file
        then redirect in_file file descriptor to STDIN

    if redirect of output to a file is requested
        redirect out_file file descriptor to STDOUT

    redirect pipe read_end to STDIN

    clean up

    run command

Parent process code:
    wait for child processes to report finished
    cleaning up

 


\end{verbatim}


\section{System Description}

Figure \ref{fig:pipes} shows the sequence of commands for the piping
setup, here with three commands, hence two pipes.
\begin{figure}
\centering
\includegraphics[width=\textwidth]{pipes.png}
\caption{\textit{This figure shows the sequence of commands for the
    piping setup, here with three commands, hence two pipes.}}
\label{fig:pipes}
\end{figure}




\section{Known Limits}
The program does not compile with the std=c11 or ansi flag set on Linux (tested on the standard gcc version shipped with ubuntu 16.04). However it does compile on OSX (clang-800.0.38).  

\section{Results}

\section{Discussion}

\section{Testing}

\subsection{Testing for closing of file descriptors}

\subsection{Testing for correct finishing of child processes}


\addcontentsline{toc}{section}{\refname}
\bibliography{references}

\end{document}
